% !TeX root = ../main.tex

\begin{resume}

1. 已经刊载的学术论文(本人是第一作者,或者导师为第一作者本人是第二作者)

\begin{publications}[before*=\vspace{1em},after*=\vspace{1em}]
  \item \textbf{Guo Jihong}, Liu Chao$^\ast$, Cao Jinfeng$^\ast$, et al. Damage identification of wind turbine blades with deep convolutional neural networks[J]. Renewable Energy, 2021, 174: 122–133. (SCI Q1 Top IF 6.274.)
  \item 曹金凤, \textbf{郭继鸿}, 李建伟, 苏天赟. 基于支持向量机的油滴识别及粒径分布特征提取算法[J]. 船海工程, 2020, 49 (02):10-14+17. (中文核心)
\end{publications}

2. 其他研究成果 % 有就写,没有就删除

\begin{achievements}[before*=\vspace{1em},after*=\vspace{1em}]
  \item 曹金凤, \textbf{郭继鸿}. 一种风力机叶片图像损伤检测和定位方法:中国, 202011101812.7 [P] (中国专利公开号)
  \item 曹金凤, \textbf{郭继鸿}, 等. 一种路桥防撞护栏:中国, 201822225813.7 [P] (中国专利公开号)
  \item \textbf{郭继鸿}, 曹金凤, 中海石油环保服务(天津)有限公司, 李祥栋, 祝凯, 等. 海底溢油图像分类识别软件系统 V1.0, 2019SR1419722. (软件著作权)
  \item 曹金凤, 纪乃华, 祝凯, 李建伟, \textbf{郭继鸿}, 等. 含油沉积物特征识别软件[简称:OSCRS] V1.0, 2019SR1420311. (软件著作权)
  \item 中海石油环保服务(天津)有限公司, 曹金凤, \textbf{郭继鸿}, 等. 海底溢油图像噪声特征提取软件[简称: OSINFE] V1.0, 2019SR1425879. (软件著作权)
\end{achievements}

3. 参与导师科研项目情况

\begin{achievements}[before*=\vspace{1em},after*=\vspace{1em}]
  \item 青岛科大科瑞信控技术有限公司,橡胶智能硫化过程温度场实时仿真系统开发,2020.7-2020.12,参与算法开发工作
  \item 自然资源部第一海洋研究所,海洋多参量数据融合技术研究与软件开发,2020.3.1-2020.10.1,参与算法与软件开发工作
  \item 清华大学,风力机叶片故障智能识别系统开发,2019.12.04-2020.11.30,参与算法开发工作
  \item 自然资源部第一海洋研究所,基于图像和底质参数的含油沉积物特征识别算法研究与软件开发,2018.12-2019.06,参与算法与软件开发工作
\end{achievements}

\end{resume}
